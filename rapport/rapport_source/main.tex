%%%%%%%%%%%%% Doing my own template
%%%%%%%%%%%%% Following the specifications of
% http://bu.univ-amu.libguides.com/c.php?g=511743&p=4025195
%%%%%%%%%%%%% Sober amu thesis template

% LAYOUT %%%%%%%%%%%%%%%%%%%%%%%%%%%%%%%%%%%%%%%%%%%%
%%% Using the report class
\documentclass[12pt, 			% Recommended
               singlespacing,         % SHOULD BE
               %singlespacing
			  ]{report}

%%% Margins
\usepackage[a4paper,
			%width=150mm,
			top=30mm,
			bottom=30mm,
            right=30mm,
            left=30mm,
            % Offset to bind pages
            bindingoffset=6mm,  
            % To make room for the header
            headheight=20mm,    
           ]{geometry}

% Line spacing - 1 is normal, 1.3 is onehalf and 1.6 is double spacing
\linespread{1}

%%% Change titles & stuff
\makeatletter
\renewcommand{\@chapapp}{Section}
\makeatother
\renewcommand{\abstractname}{Résumé}
\renewcommand{\contentsname}{Table des matières}
\renewcommand{\appendixname}{Table des annexes}

%%%%%%%%%%%%%%%%%%%%%%%%%%%%%%%%%%%%%%%%%%%%%%%%%%%%%

% Miscellaneous packages %%%%%%%%%%%%%%%%%%%%%%%%%%%%

%%% Really basic packages here to handle languages
\usepackage[english, french]{babel}
\usepackage[utf8]{inputenc}
\usepackage[T1]{fontenc}
% The font
\usepackage{concmath}
% Highlight stuff for revisions + colors and units
\usepackage{color}
\usepackage[x11names]{xcolor}
\usepackage{graphicx}
\usepackage{amsmath,amssymb, bm}
\usepackage[squaren, Gray, cdot]{SIunits}

\usepackage{listings}

% Where your pictures are
\graphicspath{ {pics/} }

%%% Header & Footer
\usepackage{fancyhdr}
\pagestyle{fancy}
\fancyhead{}
\fancyhead[RO,LE]{\rightmark}
\fancyfoot{}
\fancyfoot[LE,RO]{\thepage}
\fancyfoot[CE,CO]{\leftmark}
\fancyfoot[RE,LO]{Endless Space 2}
\renewcommand{\headrulewidth}{0.4pt}
\renewcommand{\footrulewidth}{0.4pt}

% Some symbols
\usepackage{textcomp}

% To include pdfs, like articles, directly
\usepackage{pdfpages}            

% For figures
\usepackage{caption}
\usepackage{subcaption}

% For text boxes spanning multiple pages
% And to add a color to it
\usepackage{framed}
% Access the different colors here
% http://mirrors.standaloneinstaller.com/ctan/macros/latex/contrib/xcolor/xcolor.pdf page 40
\colorlet{shadecolor}{LightSteelBlue1}
\colorlet{framecolor}{Blue4}

% New environment for shaded frames
\newenvironment{frshaded}{%
\def\FrameCommand{
\fboxrule=\FrameRule
\fboxsep=\FrameSep
\fcolorbox{framecolor}{shadecolor}
}%
\MakeFramed {\FrameRestore}}%
{\endMakeFramed}
% Use like this
%\begin{frshaded}
%\end{frshaded}

% To make todo notes, exemple in the intro
\usepackage[textwidth=3cm]{todonotes}

% For appendices
\usepackage[toc,page]{appendix}
% And change the names
\renewcommand{\appendixname}{Annexe}
\renewcommand{\appendixtocname}{Annexes}
\renewcommand{\appendixpagename}{Annexes}

% Line breaks in text
% Warning: should find something better for urls >.<
\sloppy

% Add quotes
\usepackage{dirtytalk}

% Fancy tables
\usepackage{multirow}

% Urls
\usepackage{url}

% for pdf docs to read on computers
% use this when compiling the pdf version (to send by mail)
% comment it when you want to print it
\usepackage{hyperref}

% Package only for the template
\usepackage{lipsum}

%%%%%%%%%%%%%%%%%%%%%%%%%%%%%%%%%%%%%%%%%%%%%%%%%%%%%

% Glossary %%%%%%%%%%%%%%%%%%%%%%%%%%%%%%%%%%%%%%

% if you need a glossary
% \usepackage{glossaries}         
% \makeglossaries

% \newglossaryentry{ABF}{name=ABF, description={Adaptive Biasing Force}}

% Use \newglossaryentry{utc}{name=UTC, description={Coordinated Universal Time}} to add a glossary entry within the document
% Use gls{utc} to use that entry somewhere

%%%%%%%%%%%%%%%%%%%%%%%%%%%%%%%%%%%%%%%%%%%%%%%%%%%%%

% Bibliography %%%%%%%%%%%%%%%%%%%%%%%%%%%%%%%%%%%%%%

% Required to generate language-dependent quotes in the bibliography
\usepackage[autostyle=true]{csquotes}

% Use biber - way better than bibtex
\usepackage[backend=biber,
% 			citestyle=numeric-comp,
%             bibstyle=numeric,
            citestyle=authoryear,
            bibstyle=authoryear,
            %dashed=false,
            sorting=nyt,
            natbib=true,
            doi=true,
            url=false,
            isbn=false,
            eprint=false,
            giveninits=true,
            uniquename=init,
            maxcitenames=1, 
            minbibnames=6, 
            maxbibnames=6
            ]{biblatex}

% The filename of the bibliography
\addbibresource{all.bib} 
% To escape some underscores in the .bib exported from mendeley
\DeclareSourcemap{ % Used when .bib/Bibliography is compiled, not when document is
    \maps{
        \map{ % Replaces '{\_}', '{_}' or '\_' with just '_'
            \step[fieldsource=url,
                  match=\regexp{\{\\\_\}|\{\_\}|\\\_},
                  replace=\regexp{\_}]
        }
        \map{ % Replaces '{'$\sim$'}', '$\sim$' or '{~}' with just '~'
            \step[fieldsource=url,
                  match=\regexp{\{\$\\sim\$\}|\{\~\}|\$\\sim\$},
                  replace=\regexp{\~}]
        }
        \map{ % Replaces '{\_}', '{_}' or '\_' with just '_'
            \step[fieldsource=doi,
                  match=\regexp{\{\\\_\}|\{\_\}|\\\_},
                  replace=\regexp{\_}]
        }
        \map{ % Replaces '{'$\sim$'}', '$\sim$' or '{~}' with just '~'
            \step[fieldsource=doi,
                  match=\regexp{\{\$\\sim\$\}|\{\~\}|\$\\sim\$},
                  replace=\regexp{\~}]
        }
    }
}

% Start with roman numbers until the introduction
\pagenumbering{roman}

% handle the abbverviations ?
% Basic abbreviation page here, you can play around with the two "5cm"
% To increase or decrease the spacing between the abbreviation and the words
\usepackage{calc}
\makeatletter
\newcommand{\tocfill}{\cleaders\hbox{$\m@th \mkern\@dotsep mu . \mkern\@dotsep mu$}\hfill}
\makeatother
\newcommand{\abbrlabel}[1]{\makebox[5cm][l]{\textbf{#1}\ \tocfill}}
\newenvironment{abbreviations}{\begin{list}{}{\renewcommand{\makelabel}{\abbrlabel}%
        \setlength{\labelwidth}{5cm}\setlength{\leftmargin}{\labelwidth+\labelsep}%
                                              \setlength{\itemsep}{0pt}}}{\end{list}}
                                              
% crossed text, if you need it
\usepackage[normalem]{ulem}

%%%%%%%%%%%%%%%%%%%%%%%%%%%%%%%%%%%%%%%%%%%%%%%%%%%%%

\begin{document}

%% Title page
\begin{titlepage}
\centering
\vspace*{-2cm}


\begin{figure}[htbp]
\centering
\includegraphics[width=0.3\textwidth]{logo_ensea.png}

\end{figure}
\begin{center}
	\vspace{0.4cm}
	\LARGE ENSEA\\
	\vspace{0.2cm}
	\Large École nationale supérieure de l'électronique et de ses applications\\
	\vspace{0.2cm}
	
    \begin{center}
		\vspace{1cm}
		RAPPORT DE PROJET LOGICIEL TRANSVERSAL\\
    \end{center}
	\vspace{0.4cm}
    
    \begin{center}
        \vspace{0.4cm}
        \Large QUENTIN AMIEL\\
        \Large TINGYUE TENG\\
        \Large ASLAN CHAPPE\\
        \Large HUGO THIERRY\\
        \vspace{1cm}
			
        \vspace{0.8cm}
        \large 
    \end{center}
	\vspace{1cm}
    \large \\
\end{center}

\vspace{0.2cm} \normalsize
\begin{center}
\begin{tabular}{lll}
	Christophe BARES & Professeur référent \\
    \vspace{0.1cm}
	
\end{tabular}
\end{center}
% \vspace{0.4cm}
% \begin{center}\normalsize Numéro national de thèse/suffixe local: 2018AIXM0000/000ED62\\\end{center}
\end{titlepage}


% %%% Abstract




\tableofcontents

\listoffigures



% MAIN TEXT %%%%%%%%%%%%%%%%%%%%%%%%%%%%%%%%%%%%%%%%%


% normal numbering from here
\pagenumbering{arabic}

%%%% Each chapter is in a different file
%% In the [], how the chapter will be called in the table of contents
%% In the {}, the name of the chapter in the text
%% The chaptermark is the name of the chapter in the footer
%%% You want a small enough chaptermark or it will overflow to the sides




\chapter[Présentation]{Présentation}
\chaptermark{Présentation}

\section{Archétype}

\begin{figure}[htbp]
\centering
\includegraphics[width=0.9\textwidth]{endless_space_2.jpg}
\caption[Endless Space 2]{\label{figure_simple}Endless Space 2}
\end{figure}

L’objectif de ce projet est de réaliser un jeu de type Endless Space 2. A l’origine, Endless Space 2 est un jeu vidéo de stratégie au tour par tour développé par Amplitude Studios et édité par Sega, dans lequel les joueurs incarnent une civilisation qui devra étendre son influence sur une carte, une galaxie comprenant des systèmes solaires eux-mêmes composés d’une à 4 planètes, générée aléatoirement. Les systèmes solaires font office de « villes » (si on le compare à Civilisation), ils sont reliés entre eux par des voies stellaire, qui eux servent de route. Le joueur peut se déplacer sur la galaxie à l'aide de vaisseaux spatiales militaire ou colonisateur.\\
\begin{figure}[htbp]
\centering
\includegraphics[width=0.9\textwidth]{pics/galaxie.jpg}
\caption[Exemple de galaxie]{\label{figure_simple}Exemple de galaxie}
\end{figure}
\section{Règles du jeu}

En début de partie, vous ne commencez qu’avec une planète colonisée dans l’un des nombreux systèmes stellaire du jeu. Chaque planète produira en quantité plus ou moins importante quatre ressources différentes :\\
\begin{itemize}
\item L’industrie qui servira à la construction de bâtiments et de vaisseaux,
\item La science qui permettra la rechercher de nouvelles technologies et améliorations,
\item La brume comme monnaie unique du jeu pour acheter des bâtiments ou des vaisseaux directement,
\item La nourriture pour augmenter le niveau des planètes.
\end{itemize} 

\begin{figure}[htbp]
\centering
\includegraphics[width=0.9\textwidth]{pics/système_solaire_2.jpg}
\caption[Exemple de système stellaire colonisé par une civilisation]{\label{figure_simple}Exemple de système stellaire colonisé par une civilisation}
\end{figure}

Le but du jeu est d’étendre sa civilisation à l’ensemble de la galaxie et de détruire les autres peuples. Afin de pouvoir rivaliser avec les autres peuples, il vous faudra deux choses : maîtriser l’extension de vos voisins et vous développer plus rapidement qu’eux. Ceci impose de coloniser aussi bien les planètes présentes sur votre système de départ que sur les autres systèmes présents dans la galaxie.\\


\begin{figure}[htbp]
\centering
\includegraphics[width=0.9\textwidth]{pics/vaisseau.jpg}
\caption[Exemple de vaisseau]{\label{figure_simple}Exemple de vaisseau}
\end{figure}

Pour gagner une partie d’Endless Space 2, plusieurs solutions s’offrent à vous, allant de celui qui a les plus grosses statistiques à la fin d'un nombre de tour définit, l’éradication pure et simple des autres factions ou arriver à la fin de l'arbre technologique. Côté combat, il n’est possible de se battre qu'uniquement que lorsque votre flotte est en orbite sur un système stellaire, sachant que dès qu’une route est empruntée, il est impossible de faire demi-tour tant que votre flotte n’est pas arrivée à la fin de la voie stellaire. Une fois le combat engagé, soit contre une flotte adverse, soit pour envahir une planète, ceux-ci se dérouleront de manière autonome.



\section{Ressource}

\begin{figure}[htbp]
\centering
\includegraphics[width=0.9\textwidth]{pics/spaceship_1_preview.png}
\caption[Texture pour les vaisseaux]{\label{figure_simple}Texture pour les vaisseaux}
\end{figure}

\begin{figure}[htbp]
\centering
\includegraphics[width=0.9\textwidth]{pics/planets.png}
\caption[Texture pour les planètes]{\label{figure_simple}Texture pour les planètes}
\end{figure}


%%%% Bibliography before appendices
%%%% After the initial round, you need to fix it by hand depending on where you generate your .bib file from
%%%% Mendeley often messes with italics and accents for instance


\end{document}
